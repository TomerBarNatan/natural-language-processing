(a) \newline
Let us note that due to the formula that defines each $\alpha_i$ the sum of all $\alpha_i$ is 1. 
Hence, by looking at each value as category we easily get that this is a categorical distribution. \newline
(b) \newline
this would happen if \newline 
1. there exists some $k_j$ where $k_j^Tq$ is large and \newline
2. for all other $k_i$ where $i \neq j$ $k_i^Tq$ is small. \newline
Let us remember that $a^Tb = |a||b|cos(\theta)$ where $\theta$ is the angle between $a$ and $b$. Lets us assume for simplicity sake that $|q| = 1$ and $|k_j| = 1$ for all $j$. \newline
Then, for the two conditions above to hold we need that $q$ and some single $k_j$ are almost parallel, i.e their angle is small on the unit circle. 
Whereas all other $k_i$ are almost in the opposite direction to $q$, i.e their angle with $q$ is large on the unit circle. \newline
(c) \newline
Assuming that some $k_j$ is parallel to $q$ and all other $k_i$ are in the opposite direction to $q$ we get that $k_j^Tq = 1$ and $k_i^Tq = -1$ for all $i \neq j$. \newline
We get that $e^{k_j^Tq} = e^1 = e$ and $e^{k_i^Tq} = e^{-1} = \frac{1}{e}$. \newline
We get that $\alpha_j=$



(d) \newline
This means that the output $c$ will depend almost only on the the single $v_j$ that corresponds to the $k_j$ that is parallel to $q$. 
This is, obviously, not desirable since we want the output to be contextualized, i.e depend on the full sentence. Without this property the model will not be expressive enough.